\section{Context}
\paragraph{}
This report is the result of a university assignment.
It aims to prove that the student understood the motivation, goals and means of studying pattern recognition.
Therefore, this is a way of summarizing a series of observations and experiments done on images containing shapes.

\section{Motivation}
\paragraph{}
Classifying objects is a task that we must do every day. It is incredibly versatile and universally encountered: whether we should classify a thing as being healthy or not, a shape as being a circle or not, we must do it everyday.
It is inevitable that classification tasks have made their way into Computer Science as well, in every domain.
Having accurate ways of doing these kind of tasks is, therefore, widely needed, so we will focus on analysing some methods of doing this task.

\section{Goals}
\paragraph{}
Our goal here is to analyse and classify instances from a dataset of points.
Each point is labeled, therefore this is a supervised classification task.
We will try to mathematically express relationship between the points' attributes (in our case, their positional values in the 2D space) that will help us conclude certain traits on the dataset.
The conclusions will have to be as close as possible to the ground truth, so we will also be interested in ``scoring'' as well as possible with our algorithms.