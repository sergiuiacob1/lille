\paragraph{}
When analysing covariance, we can see that: 

write about if values in sigma are positive, negative or close to 0 (see course) and exemplify with data

starting from sigma and looking at the red class and variance for the first atribute, we could that, for the first feature, the variance is low compared to other values (like 3, 5).
That would mean that the data, projected on the axis for that feature (in our case abscisa/ordonata, nu stiu sigur), would be "gathered" around.
If we look at the image, that is true!

Thinking the other way around, if we look at the figure at the green class, for example, we can see that the opposite happens for the first feature: their projections are rather dispersed.
That is, indeed, true, if we look at the co-variance matrix: the value is high: ~5.