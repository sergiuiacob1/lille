\section{Context}
\paragraph{}
This report is the result of a university assignment.
It aims to prove that the student understood the motivation, goals and means of studying pattern recognition.
Therefore, this is a way of summarizing a series of observations and experiments done on images containing shapes.

\section{Motivation}
\paragraph{}
L’objectif du TP est de segmenter une image en niveaux de gris par le seuillage automatique des niveaux de gris par des macros R que vous allez concevoir.
In order to be able to work with shapes, we must first acquire them.
Most of the time, they are not given to us in their simplest form, but found in images with all kinds of backgrounds.
They must be extracted as precisely as possible, making sure we differentiate them from the background as accurately as we can.
Where there's a need for great precision and accuracy, there's also a need for research, so in this report we'll be looking over some methods we can use to identify objects (shapes).

\section{Goals}
\paragraph{}
Having multiple grayscale images, we want to be able to do a \emph{binary segmentation} on them in order to differentiate between the objects and the image's background.
Given the fact that we only have gray values at our disposal, we must look into ways of using them in order to state that a certain pixel belongs to either an object or to the background.
