\section{Context}
\paragraph{}
This report is the result of a university assignment.
It aims to prove that the student understood the motivation, goals and means of studying pattern recognition.
Therefore, this is a way of summarizing a series of observations and experiments done on images containing shapes.

\section{Motivation}
\paragraph{}
Visualising data is a critical and essential point of analysing the data.
But we can't visualise, for example, 6D data, so it would be nice if we could \emph{reduce the dimensionality} of our problem.
Luckily, there are multiple solutions such as Principal Component Analysis that, according to \cite{pca}, help us ``convert a set of observations of possibly correlated variables into a set of values of linearly uncorrelated variables''.
We can already see advantages to this: having less variables will make visualisation possible and will also make models' training faster.

\section{Goals}
\paragraph{}
Our goal will be to successfully apply methods such as PCA on a training data set in order to reduce the space of representation for some data.
On our example, we'll work in a 2 dimensional space and we'll try to find an axis that separates our points the best.
We'll also be interested on seeing how a classification algorithm will do if we apply this pre-processing on it.