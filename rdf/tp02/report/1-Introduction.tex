\section{Context}
\paragraph{}
This report is the result of a university assignment.
It aims to prove that the student understood the motivation, goals and means of studying pattern recognition.
Therefore, this is a way of summarizing a series of observations and experiments done on images containing shapes.

\section{Motivation}
\paragraph{}
Shapes are everywhere and of many kinds. Bigger, smaller, rotated, symmetric or not and so on. But what about their contours?
How would we know how exactly to draw such a shape? This is the motivation behind trying to code and decode a \emph{shape's contour},
so one may easily reconstruct such a shape or, often times very important, \emph{simplify} them.

\section{Goals}
\paragraph{}
Suppose we have multiple shapes $S_1, \dots, S_n$, each shape $S_i$ described by the \emph{outer points} of its contour.
We need to come up with ways to code these points and also simplify them, that is to reduce the number of points.
That would allow us to reduce the memory needed to serialise these shapes, but at the cost of precision.
\label{trade-off}
Therefore, if we can somehow \emph{adjust} the trade-off between simplicity and precision,
we would have a rather versatile method to work with shape contours that could fit our needs, based on the context we find ourselves in.
